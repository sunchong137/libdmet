\documentclass[]{article}
\usepackage[left=2.5cm,right=2.5cm,top=2.5cm,bottom=2.5cm]{geometry}
\usepackage[]{amsmath}

\begin{document}
A mysterious getDir() function is used in the fitting part, we have the objective function $y=f(x)$ and gradient $g=grad(x)$, then the direction is solved by solving the linear equation $Ax=b$.
\begin{equation}
  \left( 
  \begin{array}[c]{c}
    g\\
    0.1yI
  \end{array}
  \right)
    dx
  =\left( 
  \begin{array}[c]{c}
    y\\
    0
  \end{array}
  \right)
\end{equation}
Note the coefficient matrix is $n+1$ by $n$, so the linear equations are overdetermined, and solved with least square $\min_x||b-Ax||^2$. We are not going to discuss why this direction is
chosen, but when large impurity is used, $g$ can be quite large, and the least square routine takes forever. We are here to discuss how to compute $dx$ efficiently.

The analytic expression for $dx$ is
\begin{equation}
  dx=(A^TA)^{-1}A^Tb
\end{equation}
Due to the simple form of $A$ and $b$, we can evaluate the expression. Since
\begin{equation}
  A^TA=\left( 
  \begin{array}[c]{cc}
    g^T&0.1yI
  \end{array}
  \right)\left( 
  \begin{array}[c]{c}
    g\\
    0.1yI
  \end{array}
  \right)=g^Tg+0.01y^2I=0.01y^2(100g^Tg/y^2+I)
\end{equation}
And $A^Tb=g^Ty$, we have
\begin{equation}
  dx=100y^{-2}(100g^Tg/y^2+I)^{-1}g^Ty=100(100g^Tg/y^2+I)^{-1}g^T/y=10(h^Th+I)^{-1}h^T
\end{equation}
where $h=10g/y$. Now we just need to evaluate $(h^Th+I)^{-1}$. Generally, if we can diagonalize a matrix $M=U\Lambda U^T$, then
\begin{equation}
  (M+I)^{-1}=U(I+\Lambda)^{-1}U^T
\end{equation}
can be easily evaluated. In our case, diagonalizing $h^Th$ is particularly straightforward,
\begin{equation}
  h^Th=\left( 
  \begin{array}[c]{cc}
    \tilde{h}^T&V^T
  \end{array}
  \right)\left( 
  \begin{array}[c]{cc}
    \lambda&0\\
    0&0
  \end{array}
  \right)\left( 
  \begin{array}[c]{c}
    h\\
    V
  \end{array}
  \right)
\end{equation}
where $\tilde{h}=h/||h||$ is simply the normalized vector of $h$, and $V$ the complimentary space which forms a orthogonal matrix with $\tilde{h}$. And $\lambda=||h||^2=hh^T$.
Therefore
\begin{equation}
  \begin{split}
  (h^Th+I)^{-1}&=\left( 
  \begin{array}[c]{cc}
    \tilde{h}^T&V^T
  \end{array}
  \right)\left( 
  \begin{array}[c]{cc}
    (\lambda+1)^{-1}&0\\
    0&I
  \end{array}
  \right)\left( 
  \begin{array}[c]{c}
    h\\
    V
  \end{array}
  \right)\\
  &=\left( 
  \begin{array}[c]{cc}
    \tilde{h}^T&V^T
  \end{array}
  \right)\left( 
  \begin{array}[c]{cc}
    1&0\\
    0&I
  \end{array}
  \right)\left( 
  \begin{array}[c]{c}
    h\\
    V
  \end{array}
  \right)
  -\left( 
  \begin{array}[c]{cc}
    \tilde{h}^T&V^T
  \end{array}
  \right)\left( 
  \begin{array}[c]{cc}
    1-(1+\lambda)^{-1}&0\\
    0&0
  \end{array}
  \right)\left( 
  \begin{array}[c]{c}
    h\\
    V
  \end{array}
  \right)\\
  &=I-\tilde{h}^T[1-(1+\lambda)^{-1}]\tilde{h}=I-(1+\lambda)^{-1}h^Th
  \end{split}
\end{equation}
Further, we have
\begin{equation}
  dx=10[I-(1+\lambda)^{-1}h^Th]h^T=\frac{10}{1+\lambda}h^T
\end{equation}

\end{document}
